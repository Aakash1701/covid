\documentclass[12pt,letterpaper]{article}
\setcounter{page}{0}

\usepackage{mathtools} 
\usepackage{bbm}
\usepackage[multiple]{footmisc}
\usepackage{floatpag,amsmath,amsthm,amssymb}
\newtheorem{proposition}{Proposition}
\numberwithin{equation}{section}
\newtheorem{nono-prop}{Proposition}[]

% Figure panel header font
\newcommand{\panel}{\fontfamily{phv}\selectfont\scriptsize\textbf}
\usepackage{amsmath} 
\DeclareMathOperator*{\argmin}{arg\,min}
\DeclareMathOperator*{\argmax}{arg\,max}

%%%%%%%%%%%%%%%%%%%%%%%%%%%%%%%
%% LOAD LOCAL COMPILATION PATHS
%%%%%%%%%%%%%%%%%%%%%%%%%%%%%%%

%% DON'T CHANGE ANY OF THESE PATHS. FOR LOCAL COMPILE, EDIT YOUR
%%                                            ~/include.tex ONLY
\newcommand{\HOME}{\string~}
\input{\HOME/include.tex}

% include standard package
\input{\includepath/front_matter}

\usepackage{fancyhdr} 
\pagestyle{fancy}
\lhead{}
\chead{}
\rhead{\thepage}
\cfoot{} % get rid of the page number 
\renewcommand{\headrulewidth}{0pt}
\renewcommand{\footrulewidth}{0pt}
\setlength{\headsep}{24pt}

% package for color-shared tables
\usepackage[table]{xcolor}

% disable hyperlinks, which were breaking on appendix references
% \usepackage[options]{nohyperref}

\title{COVID Comorbidity paper} \author{Nobody}

%%%%%%%%%%%%%%%%%%%%%% 
% NO TITLE PAGE
%%%%%%%%%%%%%%%%%%%%%% 
\begin{document}
\date{June 2020}
%  \maketitle

  \section{Figures and Tables}
 
  \begin{figure}[H]
    \begin{center}
      \caption{Sensitivity to Sampling Error in Hazard Ratios}
      
      \footnotesize{\textbf{A. Population Relative Risk from Health and Gender Risk
        Factors: \newline England Relative to India}}
      
      \includegraphics[scale=0.5]{\covidpath/hr_sens_prr_ratio}

      \newline
      
      \footnotesize{\textbf{B. Percent of Modeled Deaths Under Age 60: England}}
      
      \includegraphics[scale=0.5]{\covidpath/hr_sens_deaths_e}

      \newline
      
      \footnotesize{\textbf{C. Percent of Modeled Deaths Under Age 60: India}}
      
      \includegraphics[scale=0.5]{\covidpath/hr_sens_deaths_i}
    
    \end{center}
    
    \footnotesize{The figures show the sensitivity of the primary
      result to sampling error in hazard ratios. For each figure, we
      regenerated the sample statistic 1000 times, each time drawing a
      hazard ratio for each risk factor from a normal distribution
      with mean and standard deviation equal to those of the hazard
      ratios reported in [Williamson et al. 2020]. The figures show a
      histogram of the statistics generated by this exercise. Panel A
      shows the combined population relative risk from gender and all
      health conditions of mortality from COVID-19 in England divided
      by the statistic in India. The red line at 0.92 is the statistic
      reported in the paper: accounting for the distribution of health
      conditions and gender across all ages lowers modeled deaths in
      England by 8\% (a factor of 0.92) relative to India. Panels B
      and C shows the distribution of deaths under the age of 60 in
      England and India respectively from the same exercise. The red
      lines indicate the values reported in the paper.}
  \end{figure}

  \begin{figure}[H]
    \begin{center}
      \caption{Sensitivity to Sampling Error in Risk Factor Prevalence}
      
      \footnotesize{\textbf{A. Population Relative Risk from Health and Gender Risk
        Factors: \newline England Relative to India}}
      
      \includegraphics[scale=0.5]{\covidpath/prev_sens_prr_ratio}

      \newline
      
      \footnotesize{\textbf{B. Percent of Modeled Deaths Under Age 60: England}}
      
      \includegraphics[scale=0.5]{\covidpath/prev_sens_deaths_e}

      \newline
      
      \footnotesize{\textbf{C. Percent of Modeled Deaths Under Age 60: India}}
      
      \includegraphics[scale=0.5]{\covidpath/prev_sens_deaths_i}
    
    \end{center}
    
    \footnotesize{The figures show the sensitivity of the primary
      result to sampling error in risk factor prevalences. For each
      figure, we regenerated the sample statistic 1000 times, each
      time drawing a prevalence for each risk factor from a normal
      distribution with mean and standard deviation equal to those reported
      in the original data sources for each condition. The
      figures show a histogram of the statistics generated by this
      exercise. Panel A shows the combined population relative risk
      from gender and all health conditions of mortality from COVID-19
      in England divided by the statistic in India. The red line at
      0.92 is the statistic reported in the paper: accounting for the
      distribution of health conditions and gender across all ages
      lowers modeled deaths in England by 8\% (a factor of 0.92)
      relative to India. Panels B and C shows the distribution of
      deaths under the age of 60 in England and India respectively
      from the same exercise. The red lines indicate the values
      reported in the paper.}
  \end{figure}

  \begin{figure}[H]
    \begin{center}
      \caption{Sensitivity to Sampling Error in Risk Factor Prevalence}
      
      \footnotesize{\textbf{Interpolated Hazard Ratios for Age}}
      
      \includegraphics[scale=0.75]{\covidpath/age_interpolation_full}
    
    \end{center}
    
    \footnotesize{The figure shows the log hazard ratios for each age
      bin reported in OpenSAFELY (blue dashed line) and the
      interpolated values used in the analysis (solid black line). To
      interpolate the hazard ratios, we first converted them into
      natural logs, and then fitted a cubic polynomial to the
      midpoints of each bin. The graphs shows the very close fit of
      the interpolated curve to the bin means.}
  \end{figure}
  
\end{document}

